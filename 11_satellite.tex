\chapter{SATELLITE IMAGERY OF REGIONAL AIR QUALITY}

We will procure daily averages satellite images of the region for NO$_{2}$, Ozone, and SO$_{2}$ for the latest 24 months period from NASA using their Giovanni data acquisition system (\url{https://giovanni.gsfc.nasa.gov/giovanni/}).  Three different area grid sizes shall be considered with the largest covering at least a 200 X 200-km$^{2}$ area, with the others zooming in closer to the MIC and Dukhan locations.\\

Data is collected on a daily basis by the Ozone Monitoring Instrument (OMI) on the NASA Aura (EOS CH-1) satellite. 


 %
\begin{figure}[H]
\centering
\includegraphics[width=\linewidth,keepaspectratio]{images/aura.jpg} 
\caption{Aura satellite showing OMI and other sensors.}
\label{fig:aura}
\end{figure}
%


The satellite is in a 440 mile low earth orbit. Specifications of data collected for this study is shown in Table \ref{tab:omi}.

\begin{table}[H]
\centering
\caption{Pollutants measured by satellite.}
\label{tab:omi}
\begin{tabular}{@{}|l|p{3cm}p{3cm}p{3cm}@{}}
\toprule
\textbf{Pollutant} & \textbf{Ozone} & \textbf{NO$_{2}$} & \textbf{SO$_{2}$} \\ \midrule
\textbf{Measurement} & Ozone Total Column & NO$_{2}$ total column (30\% cloud screened) & SO$_{2}$ Column amount (planetary boundary) \\ \cmidrule(r){1-1}
\textbf{Units} & DU & 1/cm2 & DU \\ \cmidrule(r){1-1}
\textbf{Instrument} & OMI & OMI & OMI \\ \cmidrule(r){1-1}
\textbf{Resolution} & 0.25$^{o}$ & 0.25$^{o}$ & 0.25$^{o}$ \\ \cmidrule(r){1-1}
\textbf{Frequency} & Daily & Daily & Daily \\ \bottomrule
\end{tabular}
\end{table}


