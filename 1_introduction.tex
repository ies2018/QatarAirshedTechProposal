\chapter{INTRODUCTION}

With the rapid industrialization and urban development during the last two decades in Qatar, the need for estimation of air pollution using appropriate measurements and simulation studies with recent air dispersion software tools applicable to coastal area influence is essential to understanding the impact to human health and environmental systems.\\

Most of the major industrial establishment and operations are centered around three locations:

\begin{itemize}
\item Mesaieed Industrial City towards the south-east.
\item Ras Laffan Industrial City towards the northeast.
\item Dukhan Industrial City on the western coast represents the planned industrial operations.
\end{itemize}

The State of Qatar has established world-class heavy industrial operations within industrial cities managed by Qatar Petroleum (QP) as the local governing body. Regulatory agencies including the Ministry of Municipality and Environment (MME) and QP management, require efficient and trustworthy consultant organizations to participate in efforts to streamline environmental compliance issues and implement Ambient Air Quality Standards in Mesaieed Industrial City (MIC) and Dukhan industrial cities. \\

This proposed study includes seeks to characterize and quantify air emissions generated at the MIC and Dukhan Industrial City from stationary and mobile sources while evaluating the off-site impacts caused by dispersion of the pollutants by local meteorological patterns. The consultants will use advanced modeling techniques, historical weather and ambient air monitoring station data, best industry practices, and domain expertise to provide a comprehensive evaluation of the airsheds on these cities.
