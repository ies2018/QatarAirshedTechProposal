\chapter{PRESENTATION OF ENVIRONMENTAL CONSULTANT \& ASSOCIATES}

\section{COMPANY PROFILE (QATAR DESIGN CONSORTIUM)}

Qatar Design Consortium (QDC) is a well-established Qatari Consultancy company with Qatari as the majority shareholder. QDC has rendered independent multidisciplinary consulting services to public and private clients throughout Qatar and extended part of other Middle East countries and India for the 40 years. \\

The core business of the company lies within the spheres of Architecture, Structural and Civil Engineering, Building Services, Project Management, Environmental Engineering and Management, Energy and Utility and Management Consultancy Division. The following Figure depicts the organizational structure carried with resource asset of about 530 engineering and associated staff members as its pillars of strength. 

 %
\begin{figure}[H]
\centering
\includegraphics[width=\linewidth,keepaspectratio]{images/qdc.png} 
\caption{QDC organization.}
\label{fig:qdc}
\end{figure}
%

\subsection{ENVIRONMENTAL DIVISION ORGANIZATION}

QDC has been carrying out environmental project in association with International Consultants such as JOCOBS, Parsons Brinkerhoff among others. Since the year 2010, QDC has initiated an internal organizational strength in the field of Environmental Engineering and Management and has successfully carried out more than 100 projects on environmental management, advising, impact assessments, construction environmental management support, industrial pollution prevention and control (IPPC) and environmental permitting.

%
\begin{figure}[H]
\centering
\includegraphics[width=\linewidth,keepaspectratio]{images/qdcenv.png} 
\caption{QDC Environment Division organization.}
\label{fig:qdcenv}
\end{figure}
%

\section{PROPOSED PROJECT ORGANIZATION}

As an essential aspect of this proposal preparation, QDC has included highly competent specialist consultants, Senior Advisors and Project Team Mentor from Integrated Environmental Solutions (IES) in Kuwait; Lakes Environmental in Canada and Odour Expert to strengthen   the capabilities of project performance to exceed the line of standard set by QP at this project. In addition, QDC also proposes to associate with EXOVA Laboratory (Qatar) and Gradko Laboratory (UK) to facilitate detail profiling of VOC and Odour compounds with emphasis on mercaptans, hydrogen sulfide, ammonia, acid mist, methane and other VOCs.

\subsection{LAKES ENVIRONMENTAL}

Lakes Environmental is a Canadian company who recently celebrated its 21st year of sales in 2017 delivering advanced environmental software products, consulting and data services, and state-of-the-science environmental IT solutions. The first copy of our air quality modeling software was sold in April 1996. Lakes Environmental is the Global leading supplier of commercial air dispersion modeling software, custom IT solutions, and meteorological data sales. In addition to our highly successful commercial software products, Lakes Environmental specializes in providing  environmental IT solutions focusing on large scale air quality capabilities such as emission inventory solutions, air dispersion modeling, environmental compliance and reporting solutions, ambient monitoring analytics, human health risk assessment, and fully automated real-time and forecasting air quality modeling systems. Lakes Environmental is internationally recognized for its technologically advanced IT solutions and its exceptional expertise in atmospheric and environmental sciences.\\

Lakes Environmental’s IT solutions are in use at numerous industrial sites and government regulatory agencies throughout the world. Our emission inventory and integrated air dispersion modeling solutions are currently used by over 1,000 Tribes and local, state, and federal environmental regulatory agencies in the United States. Our web-based GIS solutions are in use at numerous state regulatory agencies including Wyoming, Nevada, Minnesota, Arizona, Hawaii, and New Jersey. In addition, Lakes Environmental was selected to provide the complete IT technology solutions for the largest ambient air monitoring system in Canada. This system is responsible for monitoring emissions and air quality associated with the extensive oil development and production regions in the Province of Alberta.\\

Furthermore, our advanced real-time and forecast modeling systems are currently in operation at multiple nuclear power production facilities located in South Korea and the Middle East. These systems are designed to provide detection, early warning, plume tracking and public safety capabilities in the event of a radiological release.\\

Lakes Environmental’s real-time and forecasting modeling systems have been deployed at Oman Oil Refineries and Petroleum Industries Company (ORPIC), located in Sohar, Oman; Glencore Sudbury Integrated Nickel Operations, located in Sudbury, Ontario, Canada; Suncor Energy, located in Sarnia, Ontario, Canada; CEZinc, located in Valleyfield, Québec, Canada; Mount Isa Mines, located in Mount Isa, Queensland, Australia; and Altos Hornos de Mexico (AHMSA), located in Monclova, Coahuila, Mexico. Primary use of these advanced real-time and forecast modeling systems is in compliance management where proactive control measures are implemented to maintain regulatory compliance, avoid exceedance events, and simultaneously maximize production opportunities. Finally, Lakes Environmental has ongoing large scale air quality project in Kuwait where our Air Quality Management Information System (AQMIS) IT solutions are being used by the Kuwait Oil Company, Kuwait Environment Public Authority (KEPA), United Nations Development Programme (UNDP) and the Kuwait Institute for Scientific Research (KISR). AQMIS is utilized by these organizations to manage all aspects of their air quality programs including emission inventory, air dispersion modeling, ambient monitoring, human health risk assessment, permit, compliance, auditing, enforcement, reporting, and mapping. Our AQMIS solution is being proposed to meet the objectives and technical scope of this project for Qatar Petroleum.\\

Lakes Environmental is a privately held company, headquartered in Waterloo Ontario Canada, with additional offices strategically located to services major industrial centers and the global regulatory community. The company is well positioned financially without any financial liability, and dedicated to heavy investment in resource acquisition and research and development efforts.

\subsection{PROJECT ROLES AND RESPONSIBILITIES}
The table below describes the identified roles involved at the project management and progress essential to ensure successful implementation and completion of the study.\\

\textbf{Dr. Jesse L. Th \'e, Project Director} \\
Prof. Jesse Th\'e is the founder and CEO of Lakes Environmental, Canada and professor of mechanical engineering at the University of Waterloo, Canada. He has over over 20 years' experience in thermal fluid flow and environmental simulations. He is the principle author of AERMOD View, CALPUFF View, IRAP-h View, and Emissions View, among others. He frequently consults for regulatory agencies in Canada, the United States, and abroad on topics related to complex air modeling projects.  With vast experience in the field of air pollutant dispersion modeling and associated software development.   \\
He will be the Project Team Mentor and provide technical guidance with the Project Coordinator, Project Manager and Senior consultants on strategic approach towards problem solution and technical knowledge resource optimization at the project. The role of Project Team Mentor as an advisor and internal project progress appraiser would play strategic role in managing the project execution and deliverables in time and in line with the set targets. His role will be available at all project stages and would lead the critical presentations.\\

\textbf{Dr. Brian Freeman, Senior Consultant} \\
Dr. Brian was the technical lead and consultant for nationwide project supported by the United Nations Development Program and Kuwait EPA to replace the current air quality compliance program with an integrated, web-based management system, identifying and quantifying over 1,000 stationary emission sources and sensitive receptors. He established Air Quality Zones for the State of Kuwait by identifying areas of microclimate and similar air dispersion patterns using CALPUFF models and earth observation satellite data.  His research correlated the observed air movements with physical borders to create clear and manageable zones that form the basis of the Kuwait Implementation Plan.
As part of the project, air quality data from stationary observations sites was collected and prepared using air zone classification procedures to provide initial determination of zones into attainment of the Kuwait Ambient Air Quality Standards and the severity of non-attainment if the zones did not meet the standards.\\
Dr Brian has conducted extensive air dispersion modeling of industrial point sources and areas sources using CALPUFF and prognostic weather data for environmental impact documentation and emergency response planning.  He has modeled facilities included flares (high and low pressure), incinerators, boilers, ground flares, open vats, fugitive dust, and landfills.	Tracks project performance of the team members consultants and takes appropriate decision and guidance. Prior to working in Kuwait, Dr Brian was an engineering officer with the US Air Force. He holds a PhD in environmental engineering from the University of Guelph in Canada and Masters of Science in Environmental and Engineering Management from the Air Force Institute of Technology at Wright-Patterson AFB, Ohio.\\
His role is to  develop a collaborative relationship with the organisation to Project Steering Board level, review method statements, develop data compilation methods and oversee modeling studies. HE will suggest appropriate mechanisms at the secondary and primary data review and suggest quality applications at monitoring. He leads quality assurances efforts with the project work, reviews  CALPUFF and SCICHEM-3 softwares application, reviews findings and conclusions of the study, coordinates dispersion modeling studies, and assists on monthly progress reviews including review of data sourcing and validations process carried out by the project team.\\


\textbf{Dr. Srinivasan M., Project Coordinator }\\
Dr. Sri has multi disciplinary educational accomplishment with specialization in environmental science and engineering.  In the twenty (28) years experience in the field of environmental engineering and management, 24 years have come in from industrial environmental engineering and management assignments. He had been technical advisor and environmental project manager for diversified industrial operations. \\
In the field of environmental management system, Dr. Srinivasan is a IRCA certified EMS auditor, registered for availability under the State of Qatar list. His expertise includes devising appropriate environmental policy for different industrial operations. He is an experienced environmental specialist both in certification process (ISO 14001) and EMS consultancy to industrial establishment for effective implementation. His other expertise includes concept to commissioning of pollution control systems. During his environmental project executions, he had established and operated Environmental analytical laboratory and has extensively carried out industrial environmental monitoring including Ambient Air Quality Assessments and Stack Emission Testing for many major chemical and petrochemical industries.
He has undergone full time training at CALPUFF and AERMOD with Lakes Environmental during the year 2012.	\\
The Project coordinator monitors and commissions the team to achieve the project deliverables. He is a key person for internal and external communications, project progress review and overall coordination between all stakeholders and the consultants.The project coordinator will be involved from the start of the project, including structuring the project strategy in conjunction with the Project Manager.  Once the project has been launched, the project coordinator will ensure that the project is actively reviewed. He Is accountable for the delivery of planned milestones associated with the project and ensures resolution of issues escalated by the Project Manager or the QP Project team. He makes key organisation/commercial decisions for the project and assures availability of essential project resources.\\

\textbf{Mr. Gulshan Mendiratta, Project Manager}
Mr Gulsham carries vast industrial monitoring and consulting experience for a period of 20 years as a part of Statutory body at the Govt. of India, Central Pollution Control Board, Ministry of Environment. He has great experiences with air pollution studies and auditing of industrial facilities compliance performances and their reporting requirements.\\

The Project Manager will be responsible for developing, in conjunction with the project coordinator, all requirements, implementation, and management of the work team. The project manager will ensure that the project is delivered on time, to budget and to the required quality standard (within agreed specifications). The project manager will ensure that the project is effectively resourced and manages relationships with a wide range of groups (including all project contributors). The Project Manager will also responsible for managing the work of consultants, allocating and utilising resources in an efficient manner and maintaining a co-operative, motivated and successful team.\\

\textbf{Dr. Suresh Kumbar, Senior Consultant – Odour Expert}\\
Dr Suresh has significant educational and professional experiences in the field of chemistry and research projects based on his research in Germany. He is an active professional working in the field of odour monitoring, employing various techniques and field equipment. He has been involved in developing odour monitoring methods, process standardization, national policy and national standards on odour for the Indian Ministry of Environment.	\\
He will lead the odour monitoring program, assess field conditions and manage the site monitoring team and the panel members, plan odour survey in consultation with stakeholders, review monitoring program, review odour control system and techniques at industrial sites, implement odour measurement campaigns, establish monitoring locations for odour assessment with odour sensing panel members, and oversee odour sampling using Tedlar bags for olfactory recordings and laboratory analysis.\\

\textbf{Dr. Manoj C., Project Data Analyst}\\
Dr. Manoj has 16 years of advanced environmental work and research with industrial, and infrastructure projects. He has successfully handled  large teams of environmental engineers and scientists for environmentally sensitive projects including the Lavasa development. \\
He will be the coordinator between the site monitoring team and client representatives, and industrial operations teams. He is responsible for primary checks on the quality at all secondary and primary data. He will oversee the transfer of relevant data and information from industrial units compliances reports and statutory CTO application which could include data and information on various pollutant sources and quantifications.\\

\textbf{Mr. B Karthikeyan, Project HSE Manager}\\
Mr. Karthikeyan holds degrees in environmental science and safety with over 25 years of professional HSE experience at various levels.\\
He will manage all HSE requirement for site visits, inspections and monitoring, provide safety training and toolbox talk to the site monitoring team, coordinate with industrial establishment HSE requirements, and report to QP HSE managers at the two regions.\\

\textbf{Dhanaraj Bharathi}, Air Quality Modeler - 1\\
Dhanaraj has a Masters in Environmental Engineering with 5 years working experiences in the field of industrial environmental and pollutant dispersion modeling including air, noise and groundwater. He has worked with air dispersion models on many projects.\\
He will lead the data management work in the preparation of all meteorological data processing, prepare parameters for CALPUFF and SCICHEM model runs, participate with the QP technical teams to review and present the progress on data sourcing,  and validate model runs.\\

\textbf{Keyur Joshi},  Air Quality Modeler -2\\

Keyur has a Masters in Industrial Pollution Control from Middlsex University and has been involved with numerous industrial EIA studies in Qatar over the last 3 years. He has undergone full time training at CALPUFF and AERMOD with Lakes Environmental. Keyur presently manages industrial environmental and pollutant dispersion modeling including air and noise levels.\\
He will assist with the data management work in the preparation of all meteorological data processing, prepare parameters for CALPUFF and SCICHEM model runs, participate with the QP technical teams to review and present the progress on data sourcing,  and validate model runs.\\

\textbf{Arvindh Somanathan, Environmental Data Analyst}\\
Arvindh has a Masters in Environmental Engineering and  has been involved with construction environmental management and  industrial EIA studies in Qatar over the last year. He has undergone full time training at CALPUFF and AERMOD with Lakes Environmental.
Arvindh carries out regular activities associated with client data sourcing and management to perform air pollutant dispersion modeling. \\
He will interact with all industrial establishments at MIC and Dukhan, in consultation with the senior advisors in the project, as well as with QP technical team. He will compile relevant data for the models at both the study locations and coordinate with all stakeholders in sourcing traffic data essential to the mobile source inputs. He will also review all industrial and associated operations associated with odour analysis phase of the study.\\

\textbf{GV Venkadesan, Monitoring Lead (Odour)}\\
GV has a Masters in Chemistry with more than 12 years of work experience in the environmental field monitoring both at industrial establishment and waste management centres. He has been performed odour and other monitoring of ambient air using handheld odour meters with criteria chemicals involving, hydrogen sulfide, ammonia, acid gases, and VOCs.\\
He will lead the monitoring team at site including the requirements at sampling with handheld monitors, collection in Tedlar bags for olfactory sensing with presence of panel members trained at the process, and coordinate sample tagging and proper transfer of samples to the laboratory analysis. \\

\textbf{S Shanmugam, Monitoring Lead (VOCs)}\\
SS is a qualified chemist and has carried out monitoring studies and environmental field monitoring both at industrial establishment and waste management centres for over 10 years. He has performed odour and monitoring of ambient air using handheld odour meters for criteria chemicals involving,  hydrogen sulfide, ammonia, acid gases,  VOCs, and landfill gases.\\
He will lead the monitoring team at the sites,  including sampling with handheld monitors, collection of samples in Tedlar bags for olfactory sensing with presence of panel members trained at the process.\\




