\chapter{BACKGROUND}

QP is strengthening their understanding of local air quality management by conducting studies to review air quality enforcement and compliance programs effectiveness. The growth of operations in industrial cities, as well as encroachment of residential areas near the industrial areas, have led to citizen complaints regarding fugitive odours and concerns of exposure to toxic chemicals from process releases.  The MME recently released the Qatar Ambient Air Quality Standards (QAAQS) that established minimum air quality standards, which the QP cities are responsible for maintaining.\\

The nature and magnitude of industrial operations in the State of Qatar indicate significant environmental concerns and the need to streamline the air quality data management and monitoring to international standards.  It is also an evident concern on the pollutant load in the ambient air that could have a significant impact on human and environmental quality. \\

The need to understand the impact of the emissions from the present load and carrying capacities within the study area is critical from a management viewpoint.\\

Ambient air quality capacity (AAQC) assessments of industrial areas are required to determine the contribution of QP operations to local air quality. AAQCs are used to evaluate annual emissions of air pollutants from operations under normal and stressed conditions, as well as evaluate the impact of long-range pollution dispersion on communities located downwind of the facilities. Individual emission sources are included to allow source apportionment studies to take place at specified receptor locations. The methodology and models used in this study can be used to determine the impact of pollution control technologies and policy changes on air quality.\\

QP initially conducted an AAQC study for Ras Laffan Industrial City (RLIC) in northern Qatar in 2015. Based on this study, QP standardized air dispersion modeling using the CALPUFF Model and identified the need to do additional studies at MIC and Dukhan industrial cities.

\section{STUDY OBJECTIVE}

The present study under the context of reference made in section 1 of the RFQ will aim to provide a comprehensive overview of the environmental performances at the two locations, with emphasis on the following components and categorized in 2 phases as in Figure \ref{fig:needs} and Figure \ref{fig:cities} below:

 %
\begin{figure}[H]
\centering
\includegraphics[width=\linewidth,keepaspectratio]{images/needs.png} 
\caption{Project objectives.}
\label{fig:needs}
\end{figure}
%

 %
\begin{figure}[H]
\centering
\includegraphics[width=\linewidth,keepaspectratio]{images/cities.png} 
\caption{Breakdown of project objections (Section 1).}
\label{fig:cities}
\end{figure}
%
 
Based on the overview of the existing industrial area operations, the Old (Salwa) Industrial Area and the New Industrial Area are not considered.  However, the Government of Qatar proposes to relocate most of the existing industrial units from Salwa Industrial area to the newly proposed QEZ-2 Special Economic Zone in the Al Karana area in the near future.\\

The need for an AAQC related to MME regulated pollutants requires the study of industrial contributions to the regional airshed and emission contributions to possible exceedances of the QAAQS.  The results from this study will be used as inputs for regulators and city  administrators to assess the potential of new projects to significantly deteriorate (or improve) local air quality.  \\

This study will evaluate the sources and emissions of several primary pollutants as shown in Table \ref{tab:pollutants}. Additionally, an ozone study will evaluate the transport and photo-transformation of precursors into ozone (O$_{3}$).


\begin{table}[H]
\centering
\caption{Air pollutants studies as part of the project}
\label{tab:pollutants}
\begin{tabular}{@{}clcc@{}}
\toprule
\textbf{POLLUTANT} & \textbf{NAME} & \textbf{TYPE} & \textbf{CAS} \\ \midrule
NOX & Nitrogen Oxides & Secondary & 10102-44-0 (NO2) \\
 &  & Primary & 10024-97-2 (NO) \\
SO$_{2}$ & Sulfur Dioxide & Primary & 7449-09-5 \\
O$_{3}$ & Ozone & Secondary & 10028-15-6 \\
PM$_{2.5}$ & PM \textless 2.5 microns & Primary & N/A \\
PM$_{10}$ & PM \textless 10 microns & Primary & N/A \\
H$_{2}$S & Hydrogen Sulfide & Primary & 7783-06-4 \\
TRS & Total Reduced Sulphur & Primary & N/A \\
VOC & Volatile Organic Components & Primary & N/A \\
NH$_{3}$ & Ammonia & Primary & 7664-41-7 \\ \bottomrule
\end{tabular}
\end{table}

This study shall take into consideration the background pollutant levels and the physical and chemical conditions that ultimately regulate important pollutant levels of industrial cities and other areas and their impact zones. 
