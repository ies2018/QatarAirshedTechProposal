\chapter{PROPOSED METHODOLOGY – SCICHEM3 DISPERSION MODEL}

SCIPUFF is a Lagrangian transport and diffusion model for atmospheric dispersion applications.  The acronym SCIPUFF stands for Second-order Closure Integrated PUFF and describes two basic aspects of the model.  First, the numerical technique employed to solve the dispersion model equations is the Gaussian puff method in which a collection of three-dimensional puffs is used to represent an arbitrary time-dependent concentration field.  Second, the turbulent diffusion parameterization used in SCIPUFF is based on the second-order turbulence closure theories, providing a direct connection between measurable velocity statistics and the predicted dispersion rates. SCIPUFF has now been expanded to include gas and aqueous phase chemistry and aerosol thermodynamics.  The reactive SCIPUFF model is referred to as SCICHEM.\\

The Lagrangian puff methodology affords a number of advantages for atmospheric dispersion applications from localized sources.  The Lagrangian scheme avoids the artificial diffusion problems inherent in any Eulerian advection scheme and allows an accurate treatment of the wide range of length scales as a plume or cloud grows from a small source size and spreads onto larger atmospheric scales. This range may extend from a few meters up to continental or global scales of thousands of kilometers. In addition, the puff method provides a very robust prediction under coarse resolution conditions, giving a flexible model for rapid assessment when detailed results are not required. The model is highly efficient for multi-scale dispersion problems since puffs can be merged as they grow and resolution is therefore adapted to each stage of the diffusion process.\\

The efficiency of SCICHEM has been improved by the implementation of adaptive time stepping and output grids.  Each puff uses a time step appropriate for resolving its local evolution rate so that the multi-scale range can be accurately described in the time domain without using a small step for the entire calculation.  The spatial output fields are also computed on an adaptive grid, avoiding the need for the user to specify grid information and providing a complete description of the concentration field within computational constraints.\\

The generality of the turbulence closure relations provides a dispersion representation for arbitrary conditions.  Empirical models based on specific dispersion data are limited in their range of application, but the fundamental relationship between the turbulent diffusion and the velocity fluctuation statistics is applicable for a much wider range.  Our understanding of the daytime planetary boundary layer velocity fluctuations provides reliable input for the second-order closure description of dispersion for these conditions. For larger scales and upper atmosphere stable conditions, the turbulence description is based on climatological information, but the closure framework is in place to accept improvement as our understanding of these regimes improves. The closure model has been applied on local scales up to 50 km range and also on continental scales up to 3,000 km range.\\

