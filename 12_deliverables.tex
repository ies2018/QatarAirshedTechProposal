\chapter{AMBIENT AIR CAPACITY STUDY DELIVERABLES}

Outputs of Ambient Air Capacity Study will be as below:\\

1.	Emission inventory for normal, abnormal and Emergency scenarios\\
2.	Model Validation \\
3.	Model results of IVE version 2.0 \\
4.	Evaluation of all background concentrations and Monitoring trends using data from AAQM stations. \\
5.	Presentation of WRF, CALMET, CALPUFF and SCICHEM Domain, grid spacing and density.\\
6.	Representation of Discrete Cartesian receptors.\\
7.	Regional, MIC and Dukhan pollution Impact Analysis will be presented through spatial variations of pollutants on aerial satellite imagery for different periods. \\
8.	Comparison of Wind flow direction in the vertical layers with at 10 cell face heights at lowest level at 20 m, 40 m, 80 m, 160 m, 320 m, 640 m, 1200 m, 2000, 3000 m, \& 4000 m above ground level (AGL) at random hours. \\
9.	Assessment of the potential benefits, if any, of additional monitoring regimes, e.g. Photochemical Assessment Monitoring Stations (PAMS). \\
10.	Assessment of NOx or VOC limited influences on Ozone formation. \\
11.	Natural vs. anthropogenic contributions to the high Particulate Matter values of the area. \\
12.	Identification of potential environmentally sensitive receptors within, and near to different sources. \\
13.	Qualitative Risk Assessment of any risks posed to those receptors of all the obtained information to highlight low, medium and high-risk sources in the state of Qatar. \\
14.	Identify the locations where the highest frequencies of exceedance occur for the various pollutants proposing possible mitigations and/or total removals of the risks through possible actions such as plant and equipment modifications, operation improvements for all kind of pollutants addressed above. \\
15.	All background concentrations will be added to the modeled ground level concentrations. \\
16.	Proposed air shed management plan as per the model findings. \\
17.	Adequacy check on the currently available AAQM network will be done. All major higher concentration spots will be identified and AAQM station will be suggested. \\
18.	All 3 scenario results will be compared with the latest available standards: 

\begin{itemize}
\item Qatar National Ambient Air Quality 
\item RLIC and MIC Ambient Air Quality Regulations/Guidelines 
\item USEPA National Ambient Air Quality Standards (NAAQS) 
\item EU Ambient Air Quality Regulations 
\item World Health Organization (WHO) standards
\item Gulf Regional Standards 
\end{itemize}

19. Highlight on likely future policy requirements and extent to which new tools might need to be developed.

\section{AIRSHED MANAGEMENT PLAN}
The objectives of the AMP are to:

\begin{itemize}
\item Protect, and where feasible, improve air quality in the Industrial city areas managed by QP.
\item Continue to raise awareness and inform the public about air quality issues in the airshed.
\item Continue to support MME in monitoring emissions of air contaminants in the respective airsheds.
\item Ensure there is no upward trend in any measured air contaminant parameters. 
\end{itemize} 


Based on the study findings, the targets for the AMP shall be developed by the Technical Committee in consultation with the MME and other stakeholders. The study shall outline the setting of targets for the AMP for respective industrial cities of QP. Targets shall be based on trends in any measured parameter. The parameters chosen shall be those used in the determination of ambient air quality objectives. 
